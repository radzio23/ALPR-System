\documentclass[12pt,a4paper]{article}

\usepackage[polish]{babel}
\usepackage[utf8]{inputenc}
\usepackage{graphicx}
\usepackage{hyperref}
\usepackage{amsmath}
\usepackage{float}
\usepackage{geometry}
\geometry{margin=2.5cm}

\title{
    \textbf{System Rozpoznawania Tablic Rejestracyjnych (ALPR)}\\
    Dokumentacja projektu
}
\author{Anna Kilińska, Urszula Nowak, Alicja Borek, Radosław Firlej}
\date{\today}

\begin{document}

\maketitle


% ---------------------------------------------------------
\section{Wprowadzenie}

Celem projektu jest stworzenie systemu ALPR (Automatic License Plate Recognition), który automatycznie wykrywa tablicę rejestracyjną na obrazie oraz odczytuje jej numer. 

% ---------------------------------------------------------
\section{Narzędzia i środowisko}

\subsection{Python}
Język programowania użyty do implementacji całego systemu.

\subsection{OpenCV}
Biblioteka do przetwarzania obrazu wykorzystywana m.in. do:
\begin{itemize}
    \item wczytywania obrazów,
    \item konwersji do skali szarości,
    \item filtracji i binaryzacji,
    \item detekcji konturów,
    \item operacji morfologicznych.
\end{itemize}
Korzystamy również z biblioteki \texttt{imutils}, która jest zestawem prostych helperów rozwiązujących typowe problemy w OpenCV.

\subsection{Haar Cascade}
Gotowy klasyfikator \texttt{haarcascade\_plate.xml} służący do detekcji tablic rejestracyjnych.  
Model został wcześniej wytrenowany metodą AdaBoost na zbiorze zdjęć tablic.

\subsection{Tesseract OCR}
Silnik OCR wykorzystywany do rozpoznawania znaków alfanumerycznych na tablicach rejestracyjnych.

% ---------------------------------------------------------
\section{Struktura projektu}

\begin{verbatim}
src/
    main.py
    config.py
    preprocessing.py
    utils.py
    haarcascade_plate.xml

data/
\end{verbatim}

\subsection{Opis modułów}

\begin{itemize}
    \item \textbf{main.py} – główny pipeline programu.
    \item \textbf{detector.py} – stałe, ścieżki i konfiguracja.
    \item \textbf{preprocessing.py} – przygotowanie ROI do OCR.
    \item \textbf{utils.py} – funkcje pomocnicze (logika poprawiania tekstu).
\end{itemize}

\section{Dane wejściowe i testowe}

System został przetestowany na zbiorze danych składającym się z 20 zdjęć pojazdów w różnych warunkach.
\begin{itemize}
    \item \textbf{Charakterystyka zdjęć}: Rozdzielczość od 1280x720 do 4K, formaty JPG i PNG.
    \item \textbf{Zróżnicowanie}: Zdjęcia obejmują pojazdy stojące na parkingu, wjeżdżające przed szlaban oraz zdjęcia wykonane w warunkach nocnych z użyciem doświetlenia.
    \item \textbf{Przygotowanie}: Zdjęcia nie wymagały wstępnej obróbki ręcznej; system automatycznie skaluje je do szerokości 800px w celu optymalizacji szybkości działania.
\end{itemize}
% ---------------------------------------------------------
\section{Opis działania systemu}

\subsection{Detekcja tablicy}
System wykorzystuje podejście Haar Cascade – szybka detekcja prostokątów o proporcjach tablicy.

\subsection{Preprocessing}
Wycięty obszar tablicy poddawany jest wstępnemu przetwarzaniu
mającemu na celu poprawę czytelności znaków przed rozpoznawaniem OCR.
Proces obejmuje m.in. konwersję do skali szarości, binaryzację oraz
operacje morfologiczne.


\subsection{OCR}
Tesseract odczytuje tekst z przygotowanego obrazu.  
Następnie:
\begin{itemize}
    \item Mapuje znaki mylone przez OCR: w wyróżniku miejsca (pierwsze 2-3 znaki) zamienia cyfry na litery (np. 0 $\to$ O), a w części numerycznej litery na cyfry (np. Z $\to$ 7).
    \item Usuwa znaki specjalne i spacje.
    \item Obsługuje specyficzne zasady dla tablic o długości powyżej 7 znaków.
    \item wynik jest oceniany punktowo (scoring).
\end{itemize}

\subsection{Wybór najlepszego wyniku}
System wybiera wynik o najwyższym score, preferując:
\begin{itemize}
    \item długość 7-8 znaków.
    \item pierwszy znak jest literą.
\end{itemize}
Zwracany jest wynik o najwyższej punktacji.

% ---------------------------------------------------------
\section{Algorytmy użyte w projekcie}

W projekcie zastosowano zestaw klasycznych algorytmów przetwarzania obrazu oraz metod rozpoznawania znaków.  
Poniżej przedstawiono wszystkie kluczowe algorytmy wykorzystane w systemie ALPR.

\subsection{Haar Cascade (detekcja obiektów)}
Do wykrywania tablic rejestracyjnych wykorzystano klasyfikator Haar Cascade zapisany w pliku 
\texttt{haarcascade\_plate.xml}.  
Algorytm opiera się na:
\begin{itemize}
    \item cechach Haar,
    \item klasyfikatorach słabych (weak classifiers),
    \item metodzie AdaBoost,
    \item kaskadowej strukturze etapów (cascade stages).
\end{itemize}
Model został wytrenowany wcześniej na zbiorze zdjęć tablic i w projekcie pełni rolę detektora obszaru tablicy.

\subsection{Binaryzacja Otsu}
Do przygotowania obrazu tablicy przed OCR wykorzystano automatyczną binaryzację metodą Otsu, która dobiera próg segmentacji minimalizując wariancję wewnątrzklasową.

\subsection{Preprocessing – Przetwarzanie wstępne} Zanim wycięty fragment tablicy (ROI) trafi do silnika OCR, jest poddawany szeregowi transformacji mających na celu maksymalizację czytelności znaków i eliminację szumów.

\begin{itemize} 
    \item \textbf{Kadrowanie (Cropping)}: Usunięcie marginesów tablicy
    \item \textbf{Skalowanie}: Podniesienie rozdzielczości obrazu do stałej wysokości 200 pikseli przy użyciu interpolacji kubicznej, co zapewnia odpowiednią wielkość znaków dla silnika Tesseract. 
    \item \textbf{Binaryzacja Otsu}: Algorytm automatycznego dobierania progu jasności na podstawie histogramu, który minimalizuje wariancję wewnątrzklasową, skutecznie oddzielając tekst od tła. 
    \item \textbf{Operacje morfologiczne}: \begin{itemize} 
    \item \textbf{Otwarcie}: usunięcie drobnych zakłóceń (szumu) z tła. 
    \item \textbf{Erozja}: pogrubienie konturów znaków, co ułatwia ich segmentację przez OCR. \end{itemize} 
    \item \textbf{Automatyczna inwersja}: Algorytm zliczający jasne piksele; jeśli tło jest ciemniejsze od znaków, obraz jest odwracany, aby zawsze dostarczyć czarny tekst na białym tle. \end{itemize}

\subsection{Algorytm poprawy tekstu i scoringu} Po surowym odczycie z silnika Tesseract, tekst jest procesowany przez funkcje pomocnicze zawarte w \texttt{utils.py}:

\begin{itemize} \item \textbf{Korekta pozycyjna}: Zamiana znaków mylonych (np. 0 na O) w wyróżniku miejsca oraz liter na cyfry (np. Z na 7) w części numerycznej tablicy. \item \textbf{System oceniania (Scoring)}: Każdy odczyt otrzymuje punkty za zgodność z polskim formatem – premiowane są wyniki o długości 7--8 znaków oraz te rozpoczynające się od litery. \end{itemize}

\subsection{OCR – Tesseract}
Rozpoznawanie znaków wykonuje silnik Tesseract OCR.  
W projekcie użyto:
\begin{itemize}
    \item trybu \texttt{--psm 8} (pojedyncze slowo),
    \item whitelist znaków (A--Z, 0--9),
    \item algorytmów segmentacji i klasyfikacji znaków wbudowanych w Tesseract.
\end{itemize}

\section{Wyniki i skuteczność}

Na podstawie przeprowadzonych testów na zbiorze 20 obrazów uzyskano następujące wyniki:

\begin{table}[H]
\centering
\begin{tabular}{|l|c|}
\hline
\textbf{Parametr} & \textbf{Wartość} \\ \hline
Poprawna detekcja obszaru tablicy & 95\% \\ \hline
Bezłędny odczyt pełnego numeru & 75\% \\ \hline
Odczyt z błędem jednego znaku & 10\% \\ \hline
Brak wykrycia / Błędny odczyt & 25\% \\ \hline
\end{tabular}
\caption{Skuteczność systemu na zbiorze testowym.}
\end{table}

\subsection{Przykładowe działanie}
\begin{itemize}
    \item \textbf{Obraz wejściowy}: \texttt{01.jpg} (tablica pod lekkim kątem).
    \item \textbf{Surowy wynik OCR}: \texttt{ASJZJNOZL}.
    \item \textbf{Wynik po korekcie}: \texttt{SJZJN07} (poprawna zamiana O $\to$ 0 oraz Z $\to$ 7 na odpowiednich pozycjach oraz usunięcie nadmiernych znaków).
    \item \textbf{Score}: 15 pkt (spełnione kryterium długości i pierwszej litery).
\end{itemize}

% ---------------------------------------------------------
\section{Instrukcja uruchomienia}
Zawarta również w pliku README.md.

\subsection{Wymagania}
\begin{itemize}
    \item Python (3.8 - 3.14)
    \item Zainstalowany Tesseract OCR
\end{itemize}

\subsection{Instalacja zależności}

\begin{verbatim}
pip install opencv-python pytesseract imutils numpy
\end{verbatim}

\subsection{Konfiguracja Tesseract}

W pliku \texttt{config.py} należy ustawić ścieżkę, w której zainstalowany jest Tesseract:

\begin{verbatim}
pytesseract.pytesseract.tesseract_cmd =
    r'C:\Program Files\Tesseract-OCR\tesseract.exe'
\end{verbatim}

\subsection{Uruchomienie programu}

\begin{verbatim}
python src/main.py
\end{verbatim}

% ---------------------------------------------------------
\section{Podział pracy}

\begin{itemize}
    \item Anna Kilińska -- implementacja kodu programu oraz pozyskanie zbioru danych.
    \item Urszula Nowak -- implementacja poprawek w kodzie i jego optymalizacja.
    \item Alicja Borek -- architektura projektu i dokumentacja.
    \item Radosław Firlej -- optymalizacja kodu i testowanie, dokumentacja.
\end{itemize}

% ---------------------------------------------------------
\section{Ograniczenia i problemy}

\begin{itemize}
    \item Haar Cascade działa słabo przy dużym kącie nachylenia tablicy.
    \item Tesseract myli niektóre znaki (np. S - 5, O - 0) oraz odczytuje znaki, których w rzeczywistości nie ma.
    \item System działa najlepiej na zdjęciach o dobrej jakości.
    \item Silne opady deszczu lub zabrudzenie fizyczne tablicy uniemożliwiają poprawną segmentację znaków
    \item Brak obsługi tablic niestandardowych.
\end{itemize}

% ---------------------------------------------------------
\section{Wnioski}

System wykazuje wysoką skuteczność przy zdjęciach robionych pod kątem prostym w dobrym oświetleniu. Dzięki zastosowaniu autorskich filtrów i logiki punktacji, udało się znacząco ograniczyć liczbę błędów typu "0 zamiast O". Głównym kierunkiem rozwoju byłoby zastąpienie klasyfikatora Haar Cascade siecią neuronową typu YOLO dla lepszej detekcji przy dużych nachyleniach tablicy. Wprowadzenie dedykowanych modeli CNN do klasyfikacji poszczególnych znaków mogłoby wyeliminować błędy. System w obecnej formie skupia się na standardowych tablicach.

% ---------------------------------------------------------

\section{Źródła i bibliografia}

\subsection{Literatura i dokumentacja}
\begin{enumerate}
    \item Dokumentacja biblioteki OpenCV (Open Source Computer Vision Library), dostęp online: \url{https://docs.opencv.org/}.
    \item Dokumentacja silnika Tesseract OCR (Google Open Source), dostęp online: \url{https://tesseract-ocr.github.io/}.
\end{enumerate}

\subsection{Zasoby techniczne i modele}
\begin{enumerate}
    \item \textbf{haarcascade\_plate.xml}: Gotowy model klasyfikatora kaskadowego wytrenowany metodą AdaBoost, służący do detekcji obiektów o cechach zbliżonych do tablic rejestracyjnych (Haar-like features).
    \url{https://github.com/spmallick/mallick_cascades/blob/master/haarcascades/haarcascade_russian_plate_number.xml?fbclid=IwY2xjawPI9lFleHRuA2FlbQIxMABicmlkETFQMmxqVGcyczRuekljWk1Qc3J0YwZhcHBfaWQQMjIyMDM5MTc4ODIwMDg5MgABHtuCtMxsZyWnDPWUksiKlUiSzhRmRBA-9vDZy6RspSJgRRdCaik_L6-mZNFM_aem_6IdAQ5cYgg8FXvMa4YN52g}
    \item \textbf{Tesseract OCR Engine}: Wykorzystany do rozpoznawania znaków alfanumerycznych z obrazów binarnych:
    \url{https://github.com/tesseract-ocr/tesseract}
    \item \textbf{Python Software Foundation}: Środowisko uruchomieniowe i biblioteki standardowe (os, re).
\end{enumerate}
\end{document}
