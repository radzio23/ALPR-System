\documentclass[12pt,a4paper]{article}

\usepackage[polish]{babel}
\usepackage[utf8]{inputenc}
\usepackage{graphicx}
\usepackage{hyperref}
\usepackage{amsmath}
\usepackage{float}
\usepackage{geometry}
\geometry{margin=2.5cm}

\title{
    \textbf{System Rozpoznawania Tablic Rejestracyjnych (ALPR)}\\
    Dokumentacja projektu
}
\author{Anna Kilińska, Urszula Nowak, Alicja Borek, Radosław Firlej}
\date{\today}

\begin{document}

\maketitle
\tableofcontents
\newpage

% ---------------------------------------------------------
\section{Wprowadzenie}

Celem projektu jest stworzenie systemu ALPR (Automatic License Plate Recognition), który automatycznie wykrywa tablicę rejestracyjną na obrazie oraz odczytuje jej numer. 

% ---------------------------------------------------------
\section{Narzędzia i środowisko}

\subsection{Python}
Język programowania użyty do implementacji całego systemu.

\subsection{OpenCV}
Biblioteka do przetwarzania obrazu wykorzystywana m.in. do:
\begin{itemize}
    \item wczytywania obrazów,
    \item konwersji do skali szarości,
    \item filtracji i binaryzacji,
    \item detekcji konturów,
    \item operacji morfologicznych.
\end{itemize}

\subsection{Haar Cascade}
Gotowy klasyfikator \texttt{haarcascade\_plate.xml} służący do detekcji tablic rejestracyjnych.  
Model został wcześniej wytrenowany metodą AdaBoost na zbiorze zdjęć tablic.

\subsection{Tesseract OCR}
Silnik OCR używany do rozpoznawania znaków.  
W projekcie zastosowano:
\begin{itemize}
    \item tryb \texttt{--psm 7} (pojedyncza linia tekstu),
    \item whitelist znaków (A--Z, 0--9),
    \item poprawki dla polskich tablic.
\end{itemize}

% ---------------------------------------------------------
\section{Struktura projektu}

\begin{verbatim}
src/
    main.py
    detector.py
    preprocessing.py
    ocr.py
    utils.py
    haarcascade_plate.xml

data/
    test_samples/
\end{verbatim}

\subsection{Opis modułów}

\begin{itemize}
    \item \textbf{main.py} – główny pipeline programu.
    \item \textbf{detector.py} – detekcja tablic (Haar + kontury).
    \item \textbf{preprocessing.py} – przygotowanie ROI do OCR.
    \item \textbf{ocr.py} – logika OCR i poprawki polskich tablic.
    \item \textbf{utils.py} – funkcje pomocnicze.
\end{itemize}

% ---------------------------------------------------------
\section{Opis działania systemu}

\subsection{1. Detekcja tablicy}
System wykorzystuje dwa podejścia:
\begin{enumerate}
    \item \textbf{Haar Cascade} – szybka detekcja prostokątów o proporcjach tablicy.
    \item \textbf{Analiza konturów} – metoda zapasowa, gdy Haar nie wykryje tablicy.
\end{enumerate}

\subsection{2. Preprocessing}
Wycięty fragment tablicy jest:
\begin{itemize}
    \item powiększany,
    \item konwertowany do skali szarości,
    \item wygładzany filtrem Gaussa,
    \item binaryzowany metodą Otsu,
    \item automatycznie odwracany, jeśli tło jest ciemne,
    \item przycinany i otaczany białą ramką.
\end{itemize}

\subsection{3. OCR}
Tesseract odczytuje tekst z przygotowanego obrazu.  
Następnie:
\begin{itemize}
    \item usuwane są znaki niealfanumeryczne,
    \item stosowane są poprawki dla polskich tablic (np. 0→O, 1→I),
    \item wynik jest oceniany punktowo (scoring).
\end{itemize}

\subsection{4. Wybór najlepszego wyniku}
System wybiera wynik o najwyższym score, preferując:
\begin{itemize}
    \item długość 4--9 znaków,
    \item dwie litery na początku,
    \item poprawny układ liter i cyfr.
\end{itemize}

% ---------------------------------------------------------
\section{Algorytmy użyte w projekcie}

W projekcie zastosowano zestaw klasycznych algorytmów przetwarzania obrazu oraz metod rozpoznawania znaków.  
Poniżej przedstawiono wszystkie kluczowe algorytmy wykorzystane w systemie ALPR.

\subsection{Haar Cascade (detekcja obiektów)}
Do wykrywania tablic rejestracyjnych wykorzystano klasyfikator Haar Cascade zapisany w pliku 
\texttt{haarcascade\_plate.xml}.  
Algorytm opiera się na:
\begin{itemize}
    \item cechach Haar,
    \item klasyfikatorach słabych (weak classifiers),
    \item metodzie AdaBoost,
    \item kaskadowej strukturze etapów (cascade stages).
\end{itemize}
Model został wytrenowany wcześniej na zbiorze zdjęć tablic i w projekcie pełni rolę detektora obszaru tablicy.

\subsection{Detekcja konturów}
Jako metoda zapasowa zastosowano klasyczne podejście:
\begin{enumerate}
    \item filtracja bilateralna,
    \item detekcja krawędzi algorytmem Canny’ego,
    \item wyszukiwanie konturów,
    \item aproksymacja konturu metodą Ramer–Douglas–Peucker,
    \item wybór konturu o proporcjach typowych dla tablicy.
\end{enumerate}

\subsection{Binaryzacja Otsu}
Do przygotowania obrazu tablicy przed OCR wykorzystano automatyczną binaryzację metodą Otsu, która dobiera próg segmentacji minimalizując wariancję wewnątrzklasową.

\subsection{OCR – Tesseract}
Rozpoznawanie znaków wykonuje silnik Tesseract OCR.  
W projekcie użyto:
\begin{itemize}
    \item trybu \texttt{--psm 7} (pojedyncza linia tekstu),
    \item whitelist znaków (A--Z, 0--9),
    \item algorytmów segmentacji i klasyfikacji znaków wbudowanych w Tesseract.
\end{itemize}

\subsection{Heurystyki i scoring}
Po odczycie tekstu zastosowano heurystyki poprawiające wynik:
\begin{itemize}
    \item zamiana cyfr na litery i odwrotnie (np. 0↔O, 1↔I),
    \item preferowanie tablic zaczynających się od dwóch liter,
    \item preferowanie długości 4--9 znaków,
    \item wybór najlepszego wyniku na podstawie punktacji.
\end{itemize}

% ---------------------------------------------------------
\section{Instrukcja uruchomienia}
Zawarta również w pliku README.md.

\subsection{Wymagania}
\begin{itemize}
    \item Python 3.8+
    \item Zainstalowany Tesseract OCR
\end{itemize}

\subsection{Instalacja zależności}

\begin{verbatim}
pip install opencv-python pytesseract imutils numpy
\end{verbatim}

\subsection{Konfiguracja Tesseract}

W pliku \texttt{ocr.py} należy ustawić ścieżkę:

\begin{verbatim}
pytesseract.pytesseract.tesseract_cmd =
    r'C:\Program Files\Tesseract-OCR\tesseract.exe'
\end{verbatim}

\subsection{Uruchomienie programu}

\begin{verbatim}
python src/main.py
\end{verbatim}

% ---------------------------------------------------------
\section{Podział pracy}

\begin{itemize}
    \item Anna Kilińska - ...
    \item Urszula Nowak - ...
    \item Alicja Borek – architektura projektu i dokumentacja.
    \item Radosław Firlej - ...
\end{itemize}

% ---------------------------------------------------------
\section{Ograniczenia i problemy}

\begin{itemize}
    \item Haar Cascade działa słabo przy dużym kącie nachylenia tablicy.
    \item Tesseract myli niektóre znaki (np. S↔5, O↔0).
    \item System działa najlepiej na zdjęciach o dobrej jakości.
    \item Brak obsługi tablic niestandardowych.
\end{itemize}

% ---------------------------------------------------------
\section{Wnioski}

Podsumowanie działania systemu, jego skuteczności oraz możliwych kierunków rozwoju (np. YOLO, CNN do OCR, lepszy preprocessing).

% ---------------------------------------------------------
\end{document}

